\documentclass[]{article}
\usepackage{lmodern}
\usepackage{amssymb,amsmath}
\usepackage{ifxetex,ifluatex}
\usepackage{fixltx2e} % provides \textsubscript
\ifnum 0\ifxetex 1\fi\ifluatex 1\fi=0 % if pdftex
  \usepackage[T1]{fontenc}
  \usepackage[utf8]{inputenc}
\else % if luatex or xelatex
  \ifxetex
    \usepackage{mathspec}
  \else
    \usepackage{fontspec}
  \fi
  \defaultfontfeatures{Ligatures=TeX,Scale=MatchLowercase}
\fi
% use upquote if available, for straight quotes in verbatim environments
\IfFileExists{upquote.sty}{\usepackage{upquote}}{}
% use microtype if available
\IfFileExists{microtype.sty}{%
\usepackage{microtype}
\UseMicrotypeSet[protrusion]{basicmath} % disable protrusion for tt fonts
}{}
\usepackage[margin=1in]{geometry}
\usepackage{hyperref}
\hypersetup{unicode=true,
            pdfborder={0 0 0},
            breaklinks=true}
\urlstyle{same}  % don't use monospace font for urls
\usepackage{color}
\usepackage{fancyvrb}
\newcommand{\VerbBar}{|}
\newcommand{\VERB}{\Verb[commandchars=\\\{\}]}
\DefineVerbatimEnvironment{Highlighting}{Verbatim}{commandchars=\\\{\}}
% Add ',fontsize=\small' for more characters per line
\usepackage{framed}
\definecolor{shadecolor}{RGB}{248,248,248}
\newenvironment{Shaded}{\begin{snugshade}}{\end{snugshade}}
\newcommand{\AlertTok}[1]{\textcolor[rgb]{0.94,0.16,0.16}{#1}}
\newcommand{\AnnotationTok}[1]{\textcolor[rgb]{0.56,0.35,0.01}{\textbf{\textit{#1}}}}
\newcommand{\AttributeTok}[1]{\textcolor[rgb]{0.77,0.63,0.00}{#1}}
\newcommand{\BaseNTok}[1]{\textcolor[rgb]{0.00,0.00,0.81}{#1}}
\newcommand{\BuiltInTok}[1]{#1}
\newcommand{\CharTok}[1]{\textcolor[rgb]{0.31,0.60,0.02}{#1}}
\newcommand{\CommentTok}[1]{\textcolor[rgb]{0.56,0.35,0.01}{\textit{#1}}}
\newcommand{\CommentVarTok}[1]{\textcolor[rgb]{0.56,0.35,0.01}{\textbf{\textit{#1}}}}
\newcommand{\ConstantTok}[1]{\textcolor[rgb]{0.00,0.00,0.00}{#1}}
\newcommand{\ControlFlowTok}[1]{\textcolor[rgb]{0.13,0.29,0.53}{\textbf{#1}}}
\newcommand{\DataTypeTok}[1]{\textcolor[rgb]{0.13,0.29,0.53}{#1}}
\newcommand{\DecValTok}[1]{\textcolor[rgb]{0.00,0.00,0.81}{#1}}
\newcommand{\DocumentationTok}[1]{\textcolor[rgb]{0.56,0.35,0.01}{\textbf{\textit{#1}}}}
\newcommand{\ErrorTok}[1]{\textcolor[rgb]{0.64,0.00,0.00}{\textbf{#1}}}
\newcommand{\ExtensionTok}[1]{#1}
\newcommand{\FloatTok}[1]{\textcolor[rgb]{0.00,0.00,0.81}{#1}}
\newcommand{\FunctionTok}[1]{\textcolor[rgb]{0.00,0.00,0.00}{#1}}
\newcommand{\ImportTok}[1]{#1}
\newcommand{\InformationTok}[1]{\textcolor[rgb]{0.56,0.35,0.01}{\textbf{\textit{#1}}}}
\newcommand{\KeywordTok}[1]{\textcolor[rgb]{0.13,0.29,0.53}{\textbf{#1}}}
\newcommand{\NormalTok}[1]{#1}
\newcommand{\OperatorTok}[1]{\textcolor[rgb]{0.81,0.36,0.00}{\textbf{#1}}}
\newcommand{\OtherTok}[1]{\textcolor[rgb]{0.56,0.35,0.01}{#1}}
\newcommand{\PreprocessorTok}[1]{\textcolor[rgb]{0.56,0.35,0.01}{\textit{#1}}}
\newcommand{\RegionMarkerTok}[1]{#1}
\newcommand{\SpecialCharTok}[1]{\textcolor[rgb]{0.00,0.00,0.00}{#1}}
\newcommand{\SpecialStringTok}[1]{\textcolor[rgb]{0.31,0.60,0.02}{#1}}
\newcommand{\StringTok}[1]{\textcolor[rgb]{0.31,0.60,0.02}{#1}}
\newcommand{\VariableTok}[1]{\textcolor[rgb]{0.00,0.00,0.00}{#1}}
\newcommand{\VerbatimStringTok}[1]{\textcolor[rgb]{0.31,0.60,0.02}{#1}}
\newcommand{\WarningTok}[1]{\textcolor[rgb]{0.56,0.35,0.01}{\textbf{\textit{#1}}}}
\usepackage{graphicx,grffile}
\makeatletter
\def\maxwidth{\ifdim\Gin@nat@width>\linewidth\linewidth\else\Gin@nat@width\fi}
\def\maxheight{\ifdim\Gin@nat@height>\textheight\textheight\else\Gin@nat@height\fi}
\makeatother
% Scale images if necessary, so that they will not overflow the page
% margins by default, and it is still possible to overwrite the defaults
% using explicit options in \includegraphics[width, height, ...]{}
\setkeys{Gin}{width=\maxwidth,height=\maxheight,keepaspectratio}
\IfFileExists{parskip.sty}{%
\usepackage{parskip}
}{% else
\setlength{\parindent}{0pt}
\setlength{\parskip}{6pt plus 2pt minus 1pt}
}
\setlength{\emergencystretch}{3em}  % prevent overfull lines
\providecommand{\tightlist}{%
  \setlength{\itemsep}{0pt}\setlength{\parskip}{0pt}}
\setcounter{secnumdepth}{0}
% Redefines (sub)paragraphs to behave more like sections
\ifx\paragraph\undefined\else
\let\oldparagraph\paragraph
\renewcommand{\paragraph}[1]{\oldparagraph{#1}\mbox{}}
\fi
\ifx\subparagraph\undefined\else
\let\oldsubparagraph\subparagraph
\renewcommand{\subparagraph}[1]{\oldsubparagraph{#1}\mbox{}}
\fi

%%% Use protect on footnotes to avoid problems with footnotes in titles
\let\rmarkdownfootnote\footnote%
\def\footnote{\protect\rmarkdownfootnote}

%%% Change title format to be more compact
\usepackage{titling}

% Create subtitle command for use in maketitle
\newcommand{\subtitle}[1]{
  \posttitle{
    \begin{center}\large#1\end{center}
    }
}

\setlength{\droptitle}{-2em}
  \title{}
  \pretitle{\vspace{\droptitle}}
  \posttitle{}
  \author{}
  \preauthor{}\postauthor{}
  \date{}
  \predate{}\postdate{}


\begin{document}

\hypertarget{results}{%
\section{Results}\label{results}}

\hypertarget{description}{%
\subsection{Description}\label{description}}

In order to illustrate some mechanisms in rethomics, we provide a simple
and reproducible example of analysis of circadian phenotype recorded
with DAM2 monitors -- a widely addopted paradigm.

TODO: * data source -- The data was obtained from \ldots{}(citation)
TODO. * description (genotypes etc) -- exhaustive description in the
\texttt{metadata.csv}

\hypertarget{data-loading}{%
\subsection{Data loading}\label{data-loading}}

First of all, the necessary \texttt{rethomics} packages are loaded.

\begin{Shaded}
\begin{Highlighting}[]
\KeywordTok{library}\NormalTok{(damr)      }\CommentTok{# input DAM2 data}
\KeywordTok{library}\NormalTok{(zeitgebr)  }\CommentTok{# periodogram computation}
\KeywordTok{library}\NormalTok{(sleepr)    }\CommentTok{# sleep analysis}
\KeywordTok{library}\NormalTok{(ggetho)    }\CommentTok{# behaviour visualisation}
\end{Highlighting}
\end{Shaded}

Then, the metadata file is read and linked to the \texttt{.txt} result
files.

\begin{Shaded}
\begin{Highlighting}[]
\NormalTok{met <-}\StringTok{ }\NormalTok{damr}\OperatorTok{::}\KeywordTok{link_dam2_metadata}\NormalTok{(}\StringTok{"./metadata.csv"}\NormalTok{, }\StringTok{"."}\NormalTok{) }\CommentTok{# linking}
\NormalTok{dt <-}\StringTok{ }\KeywordTok{load_dam2}\NormalTok{(met)                                   }\CommentTok{# loading}
\KeywordTok{summary}\NormalTok{(dt)                                            }\CommentTok{# quick summary}
\end{Highlighting}
\end{Shaded}

\begin{verbatim}
## behavr table with:
##  58  individuals
##  8   metavariables
##  2   variables
##  1.58722e+05 measurements
##  1   key (id)
\end{verbatim}

\hypertarget{preprocessing}{%
\subsection{Preprocessing}\label{preprocessing}}

We notice, from the metadata, that the two replicates do not have the
same time in baseline. We would like to express the time relative to the
important event: thetransition to \texttt{LL}. The best way is to
sustract the \texttt{baseline\_days} metavariable from the \texttt{t}
variable. This gives us an opportunity to illustrate the use
\texttt{xmv()} that maps metavariables as variables. In addition, we use
the \texttt{data.table} syntax to create, in place, a \texttt{moving}
variable. It is \texttt{TRUE} when and only when \texttt{activity} is
greater than zero:

\begin{Shaded}
\begin{Highlighting}[]
\NormalTok{dt[,t }\OperatorTok{:}\ErrorTok{=}\StringTok{ }\NormalTok{t }\OperatorTok{-}\StringTok{ }\KeywordTok{days}\NormalTok{(}\KeywordTok{xmv}\NormalTok{(baseline_days))]    }\CommentTok{# baseline sustraction. not the use of xmv}
\NormalTok{dt[, moving }\OperatorTok{:}\ErrorTok{=}\StringTok{  }\NormalTok{activity }\OperatorTok{>}\StringTok{ }\DecValTok{0}\NormalTok{]  }
\end{Highlighting}
\end{Shaded}

To simplify visualisation, we create our own \texttt{label}
metavariable, as combination of a number and \texttt{genotype}. In the
restricted context of this analysis, this acts a unique identifyier.
Importanly, we keep \texttt{id} , which is more rigourous and universal.

\begin{Shaded}
\begin{Highlighting}[]
\NormalTok{dt[, label }\OperatorTok{:}\ErrorTok{=}\StringTok{ }\KeywordTok{interaction}\NormalTok{(}\DecValTok{1}\OperatorTok{:}\NormalTok{.N, genotype), meta=T]}
\CommentTok{# print(dt)}
\end{Highlighting}
\end{Shaded}

\hypertarget{curation}{%
\subsection{Curation}\label{curation}}

It is important to see an overview of how each individual and experiment
behaved and, if necessary, alter the data accordingly. We save this
figure as Fig 3A.

\begin{Shaded}
\begin{Highlighting}[]
\CommentTok{# make a ggplot object with label on the y and moving on the z axis}
\NormalTok{fig3A <-}\StringTok{ }\KeywordTok{ggetho}\NormalTok{(dt, }\KeywordTok{aes}\NormalTok{(}\DataTypeTok{y=}\NormalTok{label, }\DataTypeTok{z=}\NormalTok{moving)) }\OperatorTok{+}\StringTok{  }
\StringTok{  }\CommentTok{# show data as a tile plot. That is z is a pixel whose intensity maps moving}
\StringTok{  }\KeywordTok{stat_tile_etho}\NormalTok{() }\OperatorTok{+}\StringTok{                  }
\StringTok{  }\CommentTok{# add layers to draw annotations to show L and D phases as white and black}
\StringTok{  }\CommentTok{# the first layer is for the baseline (until t=0)}
\StringTok{  }\KeywordTok{stat_ld_annotations}\NormalTok{(}\DataTypeTok{x_limits =} \KeywordTok{c}\NormalTok{(dt[,}\KeywordTok{min}\NormalTok{(t)], }\DecValTok{0}\NormalTok{)) }\OperatorTok{+}
\StringTok{  }\CommentTok{# in the 2nd one, we start at 0 and use grey instead of black as we work in LL}
\StringTok{  }\KeywordTok{stat_ld_annotations}\NormalTok{(}\DataTypeTok{x_limits =} \KeywordTok{c}\NormalTok{(}\DecValTok{0}\NormalTok{, dt[,}\KeywordTok{max}\NormalTok{(t)]), }\DataTypeTok{ld_colours =} \KeywordTok{c}\NormalTok{(}\StringTok{"white"}\NormalTok{, }\StringTok{"grey"}\NormalTok{))}
\end{Highlighting}
\end{Shaded}

Dead or escaped animals are falsely scored as long series of
zero-activity. Our \texttt{sleepr} packges offer a tool to detect and
remove this artifactual data:

\begin{Shaded}
\begin{Highlighting}[]
\NormalTok{dt <-}\StringTok{ }\NormalTok{sleepr}\OperatorTok{::}\KeywordTok{curate_dead_animals}\NormalTok{(dt, moving)}
\CommentTok{# make a ggplot object with label on the y and moving on the z axis}
\NormalTok{fig3B <-}\StringTok{ }\KeywordTok{ggetho}\NormalTok{(dt, }\KeywordTok{aes}\NormalTok{(}\DataTypeTok{y=}\NormalTok{label, }\DataTypeTok{z=}\NormalTok{moving)) }\OperatorTok{+}\StringTok{  }
\StringTok{    }\KeywordTok{stat_tile_etho}\NormalTok{() }\OperatorTok{+}\StringTok{                  }
\StringTok{    }\KeywordTok{stat_ld_annotations}\NormalTok{(}\DataTypeTok{x_limits =} \KeywordTok{c}\NormalTok{(dt[,}\KeywordTok{min}\NormalTok{(t)], }\DecValTok{0}\NormalTok{)) }\OperatorTok{+}
\StringTok{    }\KeywordTok{stat_ld_annotations}\NormalTok{(}\DataTypeTok{x_limits =} \KeywordTok{c}\NormalTok{(}\DecValTok{0}\NormalTok{, dt[,}\KeywordTok{max}\NormalTok{(t)]), }\DataTypeTok{ld_colours =} \KeywordTok{c}\NormalTok{(}\StringTok{"white"}\NormalTok{, }\StringTok{"grey"}\NormalTok{))}
\end{Highlighting}
\end{Shaded}

The updated version can be visualised in Fig 3B.

For the purpose of this example, we keep only individuals that have at
least five days in LL.

\begin{Shaded}
\begin{Highlighting}[]
\NormalTok{valid_dt <-}\StringTok{ }\NormalTok{dt[ , .(}\DataTypeTok{valid =} \KeywordTok{max}\NormalTok{(t) }\OperatorTok{>}\StringTok{ }\KeywordTok{days}\NormalTok{(}\DecValTok{5}\NormalTok{)), by=id]}
\NormalTok{valid_ids <-}\StringTok{ }\NormalTok{valid_dt[valid }\OperatorTok{==}\StringTok{ }\NormalTok{T, id]}
\NormalTok{dt <-}\StringTok{ }\NormalTok{dt[id }\OperatorTok\StringTok{ }\NormalTok{valid_ids]}
\KeywordTok{summary}\NormalTok{(dt)}
\end{Highlighting}
\end{Shaded}

\begin{verbatim}
## behavr table with:
##  52  individuals
##  9   metavariables
##  3   variables
##  1.40609e+05 measurements
##  1   key (id)
\end{verbatim}

Note that as a result, we now have 52 ``valid'' individuals.

\hypertarget{double-plotted-actograms}{%
\subsection{Double plotted actograms}\label{double-plotted-actograms}}

A common way of representing rythmicity in circadian experiments is to
compute ``double-plotted actograms''. In Fig S1A, we show all double
plotted actograms layed out in a grid.

\begin{Shaded}
\begin{Highlighting}[]
\NormalTok{figS1A <-}\StringTok{ }\KeywordTok{ggetho}\NormalTok{(dt, }\KeywordTok{aes}\NormalTok{(}\DataTypeTok{z =}\NormalTok{ moving), }\DataTypeTok{multiplot =} \DecValTok{2}\NormalTok{) }\OperatorTok{+}
\StringTok{            }\KeywordTok{stat_bar_tile_etho}\NormalTok{() }\OperatorTok{+}\StringTok{ }
\StringTok{            }\KeywordTok{facet_wrap}\NormalTok{( }\OperatorTok{~}\StringTok{ }\NormalTok{label, }\DataTypeTok{ncol=}\DecValTok{4}\NormalTok{) }\OperatorTok{+}
\StringTok{            }\KeywordTok{scale_y_discrete}\NormalTok{(}\DataTypeTok{name=}\StringTok{"Day"}\NormalTok{)}
\end{Highlighting}
\end{Shaded}

\hypertarget{periodograms}{%
\subsection{Periodograms}\label{periodograms}}

For each indidual, we compute a \(\chi{}^2\) periodogram and we layout
all of them in a grid (Fig S1B).

\begin{Shaded}
\begin{Highlighting}[]
\NormalTok{dt_ll <-}\StringTok{ }\NormalTok{dt[t }\OperatorTok{>}\StringTok{ }\KeywordTok{days}\NormalTok{(}\DecValTok{1}\NormalTok{)]}
\NormalTok{per_dt <-}\StringTok{ }\KeywordTok{periodogram}\NormalTok{(moving, }
\NormalTok{                        dt_ll, }
                        \DataTypeTok{resample_rate =} \DecValTok{1}\OperatorTok{/}\KeywordTok{mins}\NormalTok{(}\DecValTok{10}\NormalTok{),}
                        \DataTypeTok{FUN=}\NormalTok{chi_sq_periodogram)}

\NormalTok{per_dt <-}\StringTok{ }\KeywordTok{find_peaks}\NormalTok{(per_dt)}

\NormalTok{figS1B <-}\StringTok{ }\KeywordTok{ggperio}\NormalTok{(per_dt, }\KeywordTok{aes}\NormalTok{(}\DataTypeTok{y =}\NormalTok{ power, }\DataTypeTok{peak=}\NormalTok{peak)) }\OperatorTok{+}\StringTok{ }
\StringTok{      }\KeywordTok{geom_line}\NormalTok{() }\OperatorTok{+}
\StringTok{      }\KeywordTok{geom_line}\NormalTok{(}\KeywordTok{aes}\NormalTok{(}\DataTypeTok{y=}\NormalTok{signif_threshold), }\DataTypeTok{colour=}\StringTok{"red"}\NormalTok{) }\OperatorTok{+}\StringTok{ }
\StringTok{      }\KeywordTok{geom_peak}\NormalTok{() }\OperatorTok{+}\StringTok{ }
\StringTok{      }\KeywordTok{facet_wrap}\NormalTok{( }\OperatorTok{~}\StringTok{ }\NormalTok{label, }\DataTypeTok{ncol=}\DecValTok{4}\NormalTok{) }
\end{Highlighting}
\end{Shaded}

\begin{verbatim}
## pdf 
##   2
\end{verbatim}

\begin{verbatim}
## pdf 
##   2
\end{verbatim}


\end{document}
